\documentclass[aspectratio=43]{beamer}
\usepackage[utf8]{inputenc}
\usepackage[T1]{fontenc}
\usepackage{listings}
\title[GNOME Radio 3]{GNOME Radio 3}
\author{Ole Aamot \texttt{ole@gnome.org}}
\date{July 22th, 2020}

\usetheme{guadec}

\AtBeginSection{\frame{\sectionpage}}

\begin{document}

\begin{frame}
\titlepage
\end{frame}

\section{GNOME Radio for Debian GNU/Linux 10, Fedora Core 32 and Ubuntu 20.04}

\begin{frame}[containsverbatim]
\frametitle{GNOME Radio 3}
\framesubtitle{Free World Broadcasts}

GNOME Internet Radio Locator is Free Internet Radio Software for GNOME 3.38.

The latest GNOME Internet Radio Locator 3.0.3 release (Boston) features 125
international radio stations and 98 city map markers around the world.

\end{frame}

\begin{frame}
\frametitle{Packaging}

Debian GNU/Linux, Fedora and Ubuntu software installation packages of
GNOME Internet Radio Locator for the computer hardware architectures
i386, x86\_64 and amd64 are now available for download and installation
with the package management system tools rpm and dpkg.

\end{frame}

\begin{frame}
\frametitle{Debian GNU/Linux 10 i386}

    \begin{itemize}
        \item \url{https://www.gnome.org/~ole/debian/gnome-internet-radio-locator_3.0.0-1_i386.deb}
    \end{itemize}

\end{frame}

\begin{frame}
\frametitle{Fedora Core 32 x86\_64}

    \begin{itemize}
        \item \url{http://www.gnomeradio.org/~ole/fedora/RPMS/x86\_64/gnome-internet-radio-locator-3.0.1-1.fc32.x86\_64.rpm}
    \end{itemize}

\end{frame}

\begin{frame}
\frametitle{Ubuntu 20.04 LTS amd64}

    \begin{itemize}
        \item \url{http://www.gnomeradio.org/~ole/debian/gnome-internet-radio-locator\_3.0.2-1\_amd64.deb}
    \end{itemize}

\end{frame}

\section{Compiling from Source}

\begin{frame}
\frametitle{Compiling from Source}

    \texttt{git clone https://gitlab.gnome.org/GNOME/gnome-internet-radio-locator\\
cd gnome-internet-radio-locator\\
./autogen.sh\\
./configure\\
make\\
sudo make install\\
gnome-internet-radio-locator}

\end{frame}

\end{document}
