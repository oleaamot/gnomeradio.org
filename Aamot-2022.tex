\documentclass[UKenglish]{ifimaster}  %% ... or USenglish or norsk or nynorsk
\usepackage[utf8]{inputenc}           %% ... or latin1
\usepackage[T1]{fontenc,url}
\newenvironment{dedication}
  {%\clearpage           % we want a new page          %% I commented this
   \thispagestyle{empty}% no header and footer
   \vspace*{\stretch{1}}% some space at the top
   \itshape             % the text is in italics
   \raggedleft          % flush to the right margin
  }
  {\par % end the paragraph
   \vspace{\stretch{3}} % space at bottom is three times that at the top
   \clearpage           % finish off the page
  }
\urlstyle{sf}
\usepackage{babel,textcomp,csquotes,duomasterforside,varioref,graphicx}
\usepackage[backend=biber,style=numeric-comp]{biblatex}
%\RequirePackage{index}
%\proofmodetrue
%\newindex{xauthor}{adx}{and}{People}
%\newindex{xcmds}{cdx}{cnd}{Index of Commands and Concepts}
\usepackage{biblatex}
\usepackage{url}
\usepackage{hyperref}
\usepackage{color}
\usepackage{caption}
\usepackage[formats]{listings}
\lstloadaspects{formats}
\usepackage{makeidx}
\usepackage{courier}
\usepackage[utf8]{inputenc}
\usepackage{graphicx}
\usepackage{chicago}
\bibliography{thesis}
\inputencoding{utf8}
\usepackage{xcolor}
\definecolor{mGreen}{rgb}{0,0.6,0}
\definecolor{mGray}{rgb}{0.5,0.5,0.5}
\definecolor{mPurple}{rgb}{0.58,0,0.82}
\definecolor{backgroundColour}{rgb}{0.95,0.95,0.92}

\lstdefinestyle{CStyle}{
backgroundcolor=\color{backgroundColour},   
commentstyle=\color{mGreen},
keywordstyle=\color{magenta},
numberstyle=\tiny\color{mGray},
stringstyle=\color{mPurple},
basicstyle=\footnotesize,
breakatwhitespace=false,         
breaklines=true,                 
captionpos=b,                    
keepspaces=true,                 
numbers=left,                    
numbersep=5pt,                  
showspaces=false,                
showstringspaces=false,
showtabs=false,                  
tabsize=2,
language=C
}

\lstset{basicstyle=\footnotesize\ttfamily,breaklines=true}
\lstset{framextopmargin=50pt,frame=bottomline}
\lstset{
    language=C, % choose the language of the code
    frame=single, % adds a frame around the code
    tabsize=4, % sets default tabsize to 2 spaces
    rulesepcolor=\color{gray}
}

\author{Ole Kr. Aamot}
\title{
 Public Audio Recording Software for Recording World Sounds\\\\
 \begin{large}\\
 \\
 \\
 \\
   GNOME Gingerblue (gingerblue)\\
 \\
   \url{http://www.gingerblue.org/thesis.pdf}
  \end{large}
}
\title{\\GNOME Gingerblue 2.0.1}

\noindent\begin{small}AAMOT\,\ OLE\ 1978\-02\-20\ UNIVERSITY\ OF\ OSLO\ 2022\-02\-15\ BSc PHYS\end{small}\\
\noindent\begin{small}Public Audio Recording Software for Recording World Sounds\end{small}\\
\noindent\begin{small}GNOME Gingerblue (gingerblue) version 2.0.1 (20211025)\end{small}\\
\noindent\begin{small}http://www.gingerblue.org/thesis.pdf\end{small}

\author{Ole Kristian Aamot}
\date{15 February 2022}

\begin{document}
\duoforside[dept={Department of Physics},   %% ... or your department
  program={Electrical Engineering, Informatics and Technology},  %% ... or your programme
  short]                                        %% ... or long

\frontmatter{}
%.
%.
%.
%.
\mainmatter
\chapter*{Abstract}                   %% ... or Sammendrag or Samandrag
\noindent In this thesis I wrote Free Audio Recording Software for GTK+/GNOME.\\

GNOME Gingerblue is Free Software available under GNU General Public
License version 3 (or later) that supports immediate audio recordings in
compressed Ogg Vorbis (\url{VORBIS.COM}) encoded audio files stored in
the $\$HOME/Music/$ folder from the line input on a computer or remote
audio cards through USB connection through PipeWire
(\url{PIPEWIRE.ORG}) and WirePlumber
(\url{https://gitlab.freedesktop.org/pipewire/wireplumber}) Session
Manager via the GStreamer (\url{GSTREAMER.FREEDESKTOP.ORG})
\begin{tt}record\_pipe\end{tt} API.\\

Multiple-Location Audio Recording 1.0 is specified for recording
multiple-location audio recording configurations into Ogg Vorbis
(\url{VORBIS.COM}) compressed audio files (\url{XIPH.ORG}) in the Free
Software GNU autoconf (\url{GNU.ORG}) package GNOME Gingerblue 2.0.1
(\url{GINGERBLUE.ORG}) available under GNU General Public License
version 3 or later.\\

The Multiple-Location Audio Recording 1.0 Specification will be
implemented in GNOME Gingerblue 2.0.1 in ANSI C and available from
\url{http://WWW.GINGERBLUE.ORG/src/gingerblue-2.0.1.tar.xz} with
Source and Installation Packages for Fedora Core 35
(\url{FEDORAPROJECT.ORG}) and Ubuntu 21.04 (\url{UBUNTU.COM}).\\

The Source and Installation packages of GNOME Gingerblue 2.0.1 were
tested for recording on a Hewlett Packard laptop computer and the
MacPorts Installation package of 2.0.1 worked on Apple MacBook Air 2020
(M1) with macOS 11.6 Big Sur..\\

The Apple/HP-tested Source package of GNOME Gingerblue 2.0.1 is available from
\url{http://DOWNLOAD.GNOME.ORG/sources/gingerblue/2.0/gingerblue-2.0.1.tar.xz} and a Binary package is available for MacPorts (\url{https://www.macports.org/}) on Apple macOS (\url{https://ports.macports.org/port/gingerblue/}):

\begin{verbatim}
sudo port install gingerblue
\end{verbatim}

\chapter*{Software Implementation}

\noindent The implementation of the Multiple-Location Audio Recording
1.0 Specification (``as-is'') was completed (``as-of'') on October
25th, 2021 in C as specified in The C programming
language ({Kernighan/Ritchie, 1978}) after 21 months of work that began
on July 4th, 2018 as GNOME Gingerblue.\\

\noindent \textbf{Source Code}

\begin{itemize}
\item \url{http://www.gingerblue.org/src/gingerblue-2.0.1.tar.xz}
\item \url{https://download.gnome.org/sources/gingerblue/2.0/gingerblue-2.0.1.tar.xz}
\end{itemize}

\noindent \textbf{Fedora Core 35}

\begin{itemize}
\item \url{https://www.gingerblue.org/~ole/fedora/RPMS/x86_64/gingerblue-2.0.1-1.fc35.x86_64.rpm}
\item \url{https://www.gingerblue.org/~ole/fedora/RPMS/x86_64/gingerblue-debuginfo-2.0.1-1.fc35.x86_64.rpm}
\item \url{https://www.gingerblue.org/~ole/fedora/RPMS/x86_64/gingerblue-debugsource-2.0.1-1.fc35.x86_64.rpm}
\item \url{https://www.gingerblue.org/~ole/fedora/SRPMS/gingerblue-2.0.1-1.fc35.src.rpm}
\end{itemize}

\noindent \textbf{Ubuntu 21.04}

\begin{itemize}
\item \url{https://www.gingerblue.org/~ole/ubuntu/gingerblue_2.0.1-1_amd64.deb}
\item \url{https://www.gingerblue.org/~ole/ubuntu/gingerblue_2.0.1-1.debian.tar.xz}
\item \url{https://www.gingerblue.org/~ole/ubuntu/gingerblue_2.0.1-1.dsc}
\item \url{https://www.gingerblue.org/~ole/ubuntu/gingerblue_2.0.1-1_amd64.buildinfo}
\item \url{https://www.gingerblue.org/~ole/ubuntu/gingerblue_2.0.1-1_amd64.changes}
\item \url{https://www.gingerblue.org/~ole/ubuntu/gingerblue_2.0.1.orig.tar.xz}
\end{itemize}

\tableofcontents{}

\mainmatter{}

\chapter*{Introduksjon}                    %% ... or Introduksjon

I den første implementasjonen av grafisk lydopptaksprogramvare for
Multiple-Location Audio Recording, applikasjonen GNOME Gingerblue
versjon 2.0.1, som en applikasjon utgitt under åpen kildekode-lisens,
kan vi reprodusere lydbølger i det hørbare spekteret for menneskelige
lydopptak og lytting med tid-rom-frekvens notasjon i programmeringsspråket C ({Kernighan/Ritchie, 1978}).\\

Vi benytter prinsippene for å prosessere signalene som er motivert av
de prosessene som er involvert i lytting.\\

En representasjon av lydsignalene hvor vi har tilgang til både tid
og frekvensinformasjonen er et godt motivert valg.\\

Tids- og frekvensdomenet er et sånt domene, og det er vanligvis valgt
i lydsignalprosessering.\\

Vi ønsker imidlertid å legge til de ekstra funksjonene til
domenenavnsystemet med DNS-informasjon om vertsdatamaskinen for å
kommentere den fullstendige stedsrepresentasjonen med den unike
tid-rom-frekvensdomenerepresentasjon av hele lydsignalet i
Multiple-Location Audio Recording, motivasjonen i denne oppgaven.\\

``Å fremføre et dataprogram'' er å gjøre et lydopptak av et direktesendt radioprogram på en datamaskin ved hjelp av et direktesendt radioprogram og et lydopptaksprogram utviklet i programmeringsspråket C på en datamaskin.\\

Radiobølgene resonneres og beskrives logisk ved hjelp av dataprogrammene som kan kompileres i GCC og utføres på en datamaskin med GNU/Linux i begge oppgavene mine.\\

Opptak av Internet-radio forklarer jeg logisk i en Bachelor-oppgave gjennom radioprogrammet GNOME Internet Radio Locator (\url{http://www.gnomeradio.org/}) implementert for datamaskiner i programmeringsspråket C (\url{http://www.oleaamot.no/omu/bachelor/Aamot,2020.pdf}), gjennom lydopptaksprogrammet GNOME Gingerblue (\url{http://www.gingerblue.org/}) implementert for datamaskiner i programmeringsspråket C for det grafiske skrivebordsmiljøet GNOME (\url{http://www.gnome.org/}) beskrevet i denne oppgaven (\url{http://www.oleaamot.no/uio/bachelor/Aamot,2022.pdf}) og i en Bachelor-oppgave om GNOME Voice (\url{http://www.oleaamot.no/ntnu/bachelor/Aamot,2024.pdf}) med planlagt leveranse 24. juni 2024 ved NTNU.\\

Takk til Professor Sverre Holm ved Fysisk institutt, Universitetet i
Oslo og Dr. Wolfgang Leister ved Norsk Regnesentral som motiverte meg
til å skrive dataprogrammene GNOME Internet Radio Locator
(\url{http://www.gnomeradio.org/}) og GNOME Gingerblue
(\url{http://www.gingerblue.org/}) etter at jeg fulgte seminarene
Multimedia Coding and Transmission og Multimedia Coding and
Applications ved Norsk Regnesentral (\url{http://www.nr.no/}) i 2004
som senere ble til INF5080 og INF5081 ved Institutt for informatikk (\url{http://www.ifi.uio.no/}) ved Universitetet i Oslo (\url{http://www.uio.no/}).\\

Jeg ønsker helt til slutt å takke førsteamanuensis Arnt Inge Vistnes
som har hjulpet meg mye på veien som Bachelor-student som foreleser om
svingninger og bølger i FY-ME100 våren 2002 og senere emne FYS2130 ved
Fysisk Institutt ved Universitetet i Oslo, Knut Tomren og Tarald
Rørvik for interesse og oppmuntring, Inger Johanne Seielstad Haugli og
min tante Astrid Hanken som utvidet min interesse for lydopptak, og
min utholdende og alltid oppofrende, kjærlige Mor Gunhild Humblen og
stødige, vennlige Far Helge Aamot som gav meg min første
kassettopptaker Julen 1985 og albumet ``Face Another Day'' med The
Monroes på magnetbånd (Compact Cassette).\\\\

Endelig kan vi gjøre kontinuerlig, grafisk lydopptak under GStreamer til
lagring på solid-state drive (SSD) på en moderne datamaskin med fri Unix
som Ubuntu 21.04, Fedora 35 og macOS 11.6.\\\\

Du kan følge prosjektene på \url{http://www.gingerblue.org/} og
\url{http://wiki.gnome.org/Apps/Gingerblue} i videre utvikling.\\\\\\\\

\noindent Ole Kristian Aamot, Oslo, 15. februar 2022

\chapter*{Preface}                    %% ... or Forord

In the first Multiple-Location Audio Recording Software
implementation, the Free Software application GNOME Gingerblue version
2.0.1, as a free purpose application, we can reproduce hearable sounds
for human listening with time-space-frequency notation.

We use the principles in the processing of signals that are motivated
by the processes involved in hearing.

A representation of audio signals where we have access to both time
and frequency information is a well-motivated choice.  The
time-frequency domain is such a domain, and it is commonly deployed
in audio processing.

However, we want to add the extra capabilities of the Domain Name System
information to annotate the full location representation with the unique
time-space-frequency domain representation of the full audio signal in
Multiple-Location Audio Recording, the motivation in this thesis.

\part{Introduction}                   %% ... or Innledning or Innleiing

\chapter{Background}                  %% ... or Bakgrunn

Communication in modern day society has been greatly enhanced by mans
ability to reproduce sound.  Inventions such as telegraph, telephone,
phonograph, gramophone, radio, and later, television have benefited
from the basic concept of reproduction and preservation of the human
voice.  The act of recording therefore is best comprehended within the
context of broadcasting, telecommunication, and entertainment. ({Nmungwun, 1989})

The medium of recording rely on two components that have been the very
essence of the recording technology - magnetism and electricity.
\part{The project}                    %% ... or ??

\chapter{Planning the project}        %% ... or ??

\maketitle
\title{Multiple-Location Audio Recording 1.0}
\chapter{Historic Notes}
\section{History of Recording}

Up to the end of the 1700s, scientists had fruitlessly worked to
establish a relationship between electricity and magnetism.

In 1820, Hans Christian Ørsted, a professor at University of
Copenhagen, discovered, as mentioned in the 200 year later
non-peer-reviewed article ``Radio flux in GNOME Radio Fields
confirmed'', Aamot, Oslo Metropolitan University, 2020 --
DOI: 10.13140/RG.2.2.17889.33124 --
\url{http://www.gnomeradio.org/~ole/Aamot-2020.pdf}) and the Bachelor
thesis in Electrical Engineering (``Public Internet Radio Client for
Accessing Free Audio Maps in Countries with Free Speech'', Aamot, Oslo
Metropolitan University, 2020 -- DOI: 10.13140/RG.2.2.31344.17922 --
\url{http://www.gnomeradio.org/~ole/thesis.pdf}), that when an
electric current is passed through a wire held horizontally above a
magnetic needle that is parallel to it, the needle is deflected,
positioning itself at right angles with the conducting wire to the end
of the positive pole of the magnet.

A wire that has a constant source of electricity passed through it
becomes practically a magnet.

The tin-foil phonograph was discovered accidentally by Edison.  While
busy experimenting on a telegraphic machine (intented to repeat Morse
characters recorded on paper by indentations that transferred
messages to another circuit automatically, he stumbled upon the idea
that resulted in the phonograph.

In examining the indented paper, Edison noticed the speed at which it
moved, and a humming noise that amanated from the indentation.  This
sound was a severe rhythm almost identical to human speech hear
faintly.

In order to decipher this sound, Edison fitted a diaphragm to the
machine.  This also acred to amplify the sound.  It was then obvious
that the problem of recording human speeches and reproducing them by
mechanical means was solved.

Edison proceeded to develop a machine exclusively for capturing the
vibrations of the human voice as well as repeating them at a latter
time.  The machine was christened the ``phonograph'' (see Fig 1.).  In
November 1877, Edison officially announced his invention and on
December 24, 1877, he filed a patent application for the phonograph
with the U.S. Patent Office.  This was duly approved as patent number
200,521, issued on February 19, 1878, minus one century and one day
before February 20, 1978 (my birth date).

The tin-foil phonograph was built by John Kruesi, who had worked with
Edison for several years.  Edison had only given a rough sketch of
the phonograph to Kruesi, explaining what its functions were to be.
It was a cylinder machine, with the cylinder covered with tin-foil for
recording purposes.  When Kruesi concluded work on the machine and
brought it to Edison, he set it in motion and spoke into it:

``Mary had a little lamb,
It's fleece was white as snow.
And everywhere that Mary went,
The lamb was sure to go.''

When rewound, his exact words in clear tones were repeated, contrary
to the hoarse murmur that he anticipated, Edison was baffled at the
performance of the little machine.

Professor Joseph Henry (1797-1878) was a professor of physics at
Albany Institute whose work integrated the principles that are so much
inevitable in modern day electronics including phonographs, radio,
television and hi-fi in relation to electricity, magnetism and
mechanical energy.

Henry's theory was the basis for Morse's telegraph, Bell's telephone
and other modern sound-producing mechanisms.

His principles enabled Valdemar Poulsen to record the first sound on a
magnetized steel wire.

In 1918 a Californian, Leonard F. Fuller, had proposed the
Telegraphone wire recorder. ({BIOS, 1961})

In 1927, two U.S. Navy Research Laboratories staff members were
granted a U.S. patent for their invention, which involved the
application of high frequency (A.C.) bias to steel wire to enhance
sound reproduction.

Another Californian, James H. Alverson, proposed the use of radio
frequency to saturate steel wire in magnetic recording.  It was also
apparent that research on A.C. bias, and its use, was done under Kenzo
Nagai in Japan in the 1930s.

Three Bachelor of Science Theses written on the subject at
Massachusetts Institute of Technology in 1938 testify that magnetic
recording generated much curiosity in the late 1930s, especially in
academical circles.

By the end of October 1939 the situation had improved, with a
reduction in the use of gramophone records and more variety.

The development of radio news from Dunkirk to the end of the war can
be thought of in two parts.  The period up to D-Day and the invasion
of France in June 1944, and the rest of the war, which was then
dominated in news terms by the BBC's War Report, which provided a
day-by-day account of the final year of the war and eventual Allied
victory.  News can probably claim to be the most innovative and
successful part of BBC output at that time: ``the BBC News Department
ended the war with the most enhanced reputation and changed role of
any wartime BBC Department...''.  It began the war with just two
reports and recording equipment that required a six-ton van and had a
top speed of 20 miles per hour and ended it with coordinated coverage
of D-Day, ``a superb journalistic achievement'', with 19 reporters
using portable disc recorders and live relays heard by an audience
that reached 18 million.  There are different components of this great
transformation and these include not only improved recording
technology but the creation of a News Division, incorporating home and
empire News and Talks, under A. P. Ryan in September 1942.

It was not until April 1946 that WMAQ, (an NBC affiliate in Chicago)
aired the first completely wire-recorded news program, followed by a
competitor, WBBM (a CBS Chicago affiliate), which also deployed wire
recording for both spot-reporting and news events.

The precendent set, most network and local stations proceeded to
record their news programs on wire recorders.

In 1951, while still enjoying the fortune magnetic tape recording
implemented in audio and data recording to the recording of television
signals.

In 1951 all sound recording was on disc, but in 1952 there were six
EMI Midget recorders at Brodcasting House.

Ray Dolby, (later of audio noise reduction fame) was an exceptionally
brilliant 19-year-old high school graduate who had enrolled as an
engineering freshman at Stanford University.  Dolby dropped out of
college to join the Ampex team in August 1952.

Allthough Dolby lacked the necessary academic training in engineering,
his ingenuity and understanding of technical matters made his
contributions in the Ampex television recording project invaluable.
It was Dolby who created the basic block diagram of VTR circuitry that
is still implemented in the most recorders today.

As promising as the early efforts were, the project was again
suspended in June 1953.  In the midst of the frustration, Dolby, who
had dropped out of Stanford, was drafted into the U.S. Army and
despite friutless pleas by his colleagues he left sadly on March 18,
1953.

While he was in the Armed Forces, Dolby exchanged notes with Ginsburg.
During the project's period of official suspenson (June 1953 through
August 1954), considerable progress was made on the VTR project
despite the few man hours and the little financial allotment assigned
it, both by authorized and unauthorized means.

By 1955 tape had largely replaced the disc.  The impact of tape
recording on early current affairs broadcasting was slow to have
effect but it had the potential to solve many ``supply'' problems.

\section{History of Computing}

The Internet was not the creation of a single person.

It was the product of engineers and inventors, researchers and programmers,
and many more.  Internet prehistory was an age of ideas, and many thinkers
contributed visionary dreams that shaped what the Internet could and would
come to be.

Here are some of the pioneers:

\subsection{Vannevar Bush and his ``memex''}

Vannevar Bush was a professor in the Department of Electrical
Engineering at MIT, an influential policymaker and head of the
Carnegie Foundation, and a presidential science advisor who pushed for
government support for science.

A prolific inventor, he also designed and constructed the Differential
Analyzer, a mechanical yet sophisticated calculating device.

His 1945 ``memex'' idea foresaw how computing would allow humanity
better access to information.

\subsection{Claude Shannon and information theory}

Claude Shannon is regarded as the father of information theory.  A
1930s graduate of MIT's Master of Science and Ph.D. program, Shannon
assisted researchers with the Bush Differential Analyzer while he
completed his studies.  His astonishing realization that information
of any kind could be expressed mathematically in bits--represented by
a single zero or one--formed the basis for digital computing.

\subsection{J.C.R. Licklider and networks}

J.C.R. Licklider was associated with MIT and MIT's Lincoln Laboratory
for more than thirty years.

His articles ``The Computer as a Communication Device'' (1968) and
``Man-Computer Symbiosis'' (1960) described how interactive, networked
computers could be used for human communication, and predicted many
uses of the modern Internet.  Licklider's ideas and leadership led to
the ARPANET (Advanced Research Projects Agency Network), an early
computer network that was the original Internet.

\subsection{Bardeen/Shockley/Brattain and the transistor}

In 1956 Bell Labs scientists John Bardeen, William Shockley and Walter
Brattain shared the Nobel Prize in physics for their invention of the
transistor, a major payoff of the wartime semiconductor work,
according to Robert Buderi's book ``The INVENTION That CHANGED the
WORLD'' (Simon & Schuster, New York, 1996).

The three met just after World War II, when Bell Labs charged Shockley
with the job of building a solid state amplifier.

The Magic Month, actually a five-week span that saw the birth of the
transistor and the genesis of two Nobel Prizes, opened on November 17, 1947.

Walter Brattain had been purusing the team's goal of building the base of
fundamental knowledge and testing the surface-state theory.

On November 22, 1947, the Saturday before Thanksgiving, John Bardeen summarized much of the work while filling seven pages in his notebook.  He concluded, ``...these tests show definitely that it is possible to introduce an electrode or grid to control the flow of current in a semiconductor.''

In December 1956 the Nobel Prizes in Physics were granted to the Bell Labs colleagues Bardeen, Brattain and Shockley.

\subsection{Tim Berners-Lee and the Web}

In 1989 Tim Berners-Lee, Professor at MIT's Computer Science and
Artificial Intelligence Laboratory, invented the World Wide Web
at the CERN Lab in Switzerland and directed his work toward the
W3C Consortium (\url{http://www.w3.org/}), the Web Research Institute
and the World Wide Web Foundation.

These organizations study the future use and design of the Web, make
recommendations about technical standards, and implement projects
designed to realize the full potential of the World Wide Web.

\subsection{Philippe Defert and httpd}

Philippe Defert (1954 - 2013) working at the CERN IT Department in
Switzerland made it possible through his work on the httpd server to
publish widespread information to reach millions of people like AM/FM
radio previously did, but without global censorship, except by
ICANN.org, by decentralized domain name registrars and individual
domain holders.

\section{History of Domain Name System}

Paul Mockapetris expanded the Internet beyond its academic origins
by inventing the Domain Name System (DNS) in 1983.
 
Previously computers connected to the Internet were addressed with
IP addresses, not resolvable by domain names.

But the invention of the DNS in 1983 and the original Internet
Standards in 1986 after the creation of the Internet Engineering Task
Force IETF made this possible.

The two documents that marked the start are RFC 1034 and RFC
1035. They describe the whole protocol functionality and include data
types that it can carry.

The latest version of the Internet software Berkeley Internet Name
Domain 9 (``bind'' and ``named'') by the Internet Software Consortium
helped bring the Domain Name System to the entire world in 2000.

\chapter{Hardware}

Legitimate audio can be originated in a number of ways.  Until recent
advances in digital technology, a musician's options were limited to
getting low-grade audio from distant sources or filing reports on
location over ordinary telephones, or travelling to and from the
location with a microphone, lead and portable tape recorder, most
commonly a Uher.  Despite the great weight of the Uher, whose strap
gave so many reporters shoulder strain, it is fondly remembered for
the acceptable compromise it represented between ease of use,
broadcast quality and portability.  But many have welcomed the later
generations of hardware, including Digital Audio Tape (DAT), MiniDisc
and hard disk recorders.  For ease of use, portability and
concealability (in situations where, for reasons of personal safety,
musicians might prefer to blend into their surroundings), the recent
developed microphones that record on to a chip housed within their own
stem, and an associated USB port for downloading the audio as data
into any computer, are attrative alternatives.

Digital technology has also facilitated the establishment of live
connections between the news desk and remote locations.  Expensive,
fixed, broadcast-quality analogue landlines and often hard-to-set-up
VHF radio links from outside broadcast vehicles have, for the moment,
been eclipsed by ISDN lines, which deploy digital/analogue converters
(codecs), satellite uplinking and the Internet.

\chapter{Software}

\section{World Wide Web}

The release of Apache HTTP Server (\url{http://httpd.apache.org/}) by
the Apache Software Consortium and the release of PHP Programming
Language (\url{https://www.php.net}) and Wordpress Weblog Software
(\url{https://www.wordpress.org/}) is essential in human's ability to
reach millions of people on the World Wide Web.

\section{GNOME}

The release of the GNOME desktop software in 1997 made it possible for
humans to interactively access and store information on a computer.

\section{GStreamer, PipeWire and WirePlumber}

The release of the GStreamer software in 2004 marked the future
for multimedia on GNU/Linux and other desktop platforms running
GNOME.

Control flow in the program is determined by conditional statements.
The outcome of such tests controls the further flow of the program.

GStreamer is the software for audio recording and playback, signaling
and control in Free Desktops such as GNOME.

For example, by using conditional statements, control function values
can be reset and instruments can turn themselves on or off or be
instructed to influence one another.

\section{GNOME Gingerblue}

GNOME Gingerblue completes the task of recording live audio streams on
any computer that runs a Unix-compatible GNU system with the Linux
kernel or a Apple macOS system with Macports.org and lets you
recording/download audio into a laptop that can be edited on the site,
saved as an Ogg Vorbis with XML meta data information, and perhaps
using other developments in mobile phone technology, sent over the
Internet in a fraction of the time it once took musicians to return to
base and then edit it using traditional techniques.

GNOME Gingerblue 2.0.1 is available and builds/runs on GNOME 41
systems such as Fedora Core 35.

It supports immediate, live audio recording in compressed Xiph.org Ogg
Vorbis encoded audio files stored in the private $\$HOME/Music/$
directory from the microphone/input line on a computer or remote audio
cards through USB connection through PipeWire
(\url{http://WWW.PIPEWIRE.ORG/}) with GStreamer
(\url{http://GSTREAMER.FREEDESKTOP.ORG/}) on Fedora Core 34
(\url{https://GETFEDORA.ORG/}).

In GNOME Gingerblue version 2.0.1, the first implementation of
Multiple-Location Audio Recording, as published in the thesis, audio
and control rates are implemented by separate loops.

When composing with the computer, sounds are recorded digitally with a
sampling rate of at least 40,000 Hz and an amplitude resolution of at
least 16 bits.

The audio signals recorded with GNOME Gingerblue version 2.0.1 have a
sample rate of 44,100 Hz.

See the GNOME Gingerblue project (\url{https://WWW.GINGERBLUE.ORG/})
for screenshots, Fedora Core 34 x86\_64 RPM package and GNU autoconf
installation package
(\url{https://DOWNLOAD.GNOME.ORG/sources/gingerblue/2.0/gingerblue-2.0.1.tar.xz})
for GNOME 41 systems and
\url{https://GITLAB.GNOME.ORG/ole/gingerblue.git} for the GPLv3 source
code in my GNOME Git repository.

{\includegraphics[scale=3.0]{gingerblue-140-001.png}}

Gingerblue music recording session screen. Click “Next” to begin session.

{\includegraphics[scale=3.0]{gingerblue-140-002.png}}

The default name of the musician is extracted from g\_get\_real\_name(). You can edit the name of the musician and then click “Next” to continue ((or “Back” to start all over again) or “Finish” to skip the details).

{\includegraphics[scale=3.0]{gingerblue-140-003.png}}

Type the name of the musical song name. Click “Next” to continue ((or “Back” to start all over again) or “Finish” to skip any of the details).

{\includegraphics[scale=3.0]{gingerblue-140-004.png}}

Type the name of the musical instrument. The default instrument is “Guitar”. Click “Next” to continue ((or “Back” to start all over again) or “Finish” to skip any of the details).

{\includegraphics[scale=3.0]{gingerblue-140-005.png}}

Type the name of the audio line input. The default audio line input is “Mic” ( gst\_pipeline\_new("record\_pipe") in GStreamer). Click “Next” to continue ((or “Back” to start all over again) or “Finish” to skip the details).

{\includegraphics[scale=3.0]{gingerblue-140-006.png}}

Enter the recording label. The default recording label is “GNOME” (Free label). Click “Next” to continue ((or “Back” to start all over again) or “Finish” to skip the details).

{\includegraphics[scale=3.0]{gingerblue-140-007.png}}

Enter the Computer. The default station label is a Fully-Qualified Domain Name (g\_get\_host\_name()) for the local computer. Click “Next” to continue ((or “Back” to start all over again) or “Finish” to skip the details).

{\includegraphics[scale=3.0]{gingerblue-140-008.png}}

Notice the immediate, live recording file.  The default immediate, live recording file name falls back to the result of\\
g\_strconcat(g\_get\_user\_special\_dir(G\_USER\_DIRECTORY\_MUSIC), "/",\\
gtk\_entry\_get\_text(GTK\_ENTRY(musician\_entry)),\\
"\_-\_", gtk\_entry\_get\_text(GTK\_ENTRY(song\_entry)),\\
"\_[",g\_date\_time\_format\_iso8601 (datestamp),"]",".ogg",\\
NULL) in gingerblue/src/gingerblue-main.c

{\includegraphics[scale=3.0]{gingerblue-140-009.png}}

Studio configuration resolves the server address of your local computer.

{\includegraphics[scale=3.0]{gingerblue-140-010.png}}

Album configuration is the playlist of the compilation of multiple audio files.

{\includegraphics[scale=3.0]{gingerblue-140-011.png}}

Broadcasting to the World Wide Web (a Wordpress Webblog installation)
is the step after recording your audio files.

Click on “Cancel” once in GNOME Gingerblue to stop immediate recording and click on “Cancel” once again to exit the application (or Ctrl-c in the terminal).

\newpage

The following Multiple-Location Audio Recording XML file [.gingerblue] is created in G\_USER\_DIRECTORY\_MUSIC (usually $\$HOME/Music/$ on American English systems):

\begin{scriptsize}
\begin{verbatim}
  <?xml version='1.0' encoding='UTF-8'?>
  <gingerblue version='2.0.1'>
    <musician>Wilber</musician>
    <song>Gingerblue Track 0001</song>
    <instrument>Piano</instrument>
    <line>Mic</line>
    <label>GNOME Music</label>
    <station>streaming.gnome.org</station>
    <filename>/home/wilber/Music/Wilber_-_Song_-_2021-07-12T21:36:07.624570Z.ogg</filename>
  </gingerblue>
\end{verbatim}
\end{scriptsize}

You’ll find the recorded Ogg Vorbis audio files along with the Multiple-Location Audio Recording XML files in\\
g\_get\_user\_special\_dir(G\_USER\_DIRECTORY\_MUSIC) (usually $\$HOME/Music/$) on GNOME 41 systems configured in the American English language.

In GNOME Gingerblue version 2.0.1, the first implementation of
Multiple-Location Audio Recording, as published in the thesis, audio
and control rates are implemented by separate loops.

When composing with the computer, sounds are recorded digitally with a
sampling rate of at least 40,000 Hz and an amplitude resolution of at
least 16 bits.

The audio signals recorded with GNOME Gingerblue version 2.0.1 have a
sample rate of 44,100 Hz and are stored in the $\$HOME/Music/$ folder.

\chapter{Internet}
\section{Apache HTTP}

The Apache HTTP server is available for Unix-platforms such as Debian,
Fedora and Ubuntu from \url(http://httpd.apache.org/}

\section{PHP}

The programming language PHP is available from \url{http://www.php.net/} and is availble as a
Apache module for the Apache HTTP Server \url{http://httpd.apache.org/}.

\section{Wordpress}

The World Wide Web Blog software Wordpress is available from \url{http://www.wordpress.org/}

\chapter{Source}
\section{gingerblue 2.0.1}

\lstset{ numbers=left, stepnumber=1, firstnumber=1, numberfirstline=true }

\subsection{gingerblue-2.0.1/src/gingerblue-chord.h}

\begin{scriptsize}
  
  \lstinputlisting[format=C]{gingerblue-2.0.1/src/gingerblue-chord.h}

\end{scriptsize}
    
\subsection{gingerblue-2.0.1/src/gingerblue-config.h}

\begin{scriptsize}
  
  \lstinputlisting[format=C]{gingerblue-2.0.1/src/gingerblue-config.h}
  
\end{scriptsize}

\subsection{gingerblue-2.0.1/src/gingerblue-file.h}

\begin{scriptsize}
  
  \lstinputlisting[format=C]{gingerblue-2.0.1/src/gingerblue-file.h}
  
\end{scriptsize}

\subsection{gingerblue-2.0.1/src/gingerblue.h}

\begin{scriptsize}
  
  \lstinputlisting[format=C]{gingerblue-2.0.1/src/gingerblue.h}
  
\end{scriptsize}

\subsection{gingerblue-2.0.1/src/gingerblue-knob.h}

\begin{scriptsize}
  
  \lstinputlisting[format=C]{gingerblue-2.0.1/src/gingerblue-knob.h}
  
\end{scriptsize}

\subsection{gingerblue-2.0.1/src/gingerblue-line.h}

\begin{scriptsize}
  
  \lstinputlisting[format=C]{gingerblue-2.0.1/src/gingerblue-line.h}
  
\end{scriptsize}

\subsection{gingerblue-2.0.1/src/gingerblue-main.h}

\begin{scriptsize}
  
  \lstinputlisting[format=C]{gingerblue-2.0.1/src/gingerblue-main.h}
  
\end{scriptsize}

\subsection{gingerblue-2.0.1/src/gingerblue-main-loop.h}

\begin{scriptsize}
  
  \lstinputlisting[format=C]{gingerblue-2.0.1/src/gingerblue-main-loop.h}
  
\end{scriptsize}

\subsection{gingerblue-2.0.1/src/gingerblue-record.h}

\begin{scriptsize}
  
  \lstinputlisting[format=C]{gingerblue-2.0.1/src/gingerblue-record.h}
  
\end{scriptsize}

\subsection{gingerblue-2.0.1/src/gingerblue-song.h}

\begin{scriptsize}
  
  \lstinputlisting[format=C]{gingerblue-2.0.1/src/gingerblue-song.h}
  
\end{scriptsize}

\subsection{gingerblue-2.0.1/src/gingerblue-studio-config.h}

\begin{scriptsize}
  
  \lstinputlisting[format=C]{gingerblue-2.0.1/src/gingerblue-studio-config.h}

\end{scriptsize}

\subsection{gingerblue-2.0.1/src/gingerblue-studio-stream.h}

\begin{scriptsize}
  
  \lstinputlisting[format=C]{gingerblue-2.0.1/src/gingerblue-studio-stream.h}
  
\end{scriptsize}

\subsection{gingerblue-2.0.1/src/gingerblue-app.c}

\begin{scriptsize}
  
  \lstinputlisting[format=C]{gingerblue-2.0.1/src/gingerblue-app.c}
  
\end{scriptsize}

\subsection{gingerblue-2.0.1/src/gingerblue.c}

\begin{scriptsize}
  
  \lstinputlisting[format=C]{gingerblue-2.0.1/src/gingerblue.c}
  
\end{scriptsize}

\subsection{gingerblue-2.0.1/src/gingerblue-config.c}

\begin{scriptsize}
  
  \lstinputlisting[format=C]{gingerblue-2.0.1/src/gingerblue-config.c}
  
\end{scriptsize}

\subsection{gingerblue-2.0.1/src/gingerblue-file.c}

\begin{scriptsize}
  
  \lstinputlisting[format=C]{gingerblue-2.0.1/src/gingerblue-file.c}
  
\end{scriptsize}

\subsection{gingerblue-2.0.1/src/gingerblue-knob.c}

\begin{scriptsize}
  
  \lstinputlisting[format=C]{gingerblue-2.0.1/src/gingerblue-knob.c}
  
\end{scriptsize}

\subsection{gingerblue-2.0.1/src/gingerblue-line.c}

\begin{scriptsize}
  
  \lstinputlisting[format=C]{gingerblue-2.0.1/src/gingerblue-line.c}
  
\end{scriptsize}

\subsection{gingerblue-2.0.1/src/gingerblue-main.c}

\begin{scriptsize}
  
  \lstinputlisting[format=C]{gingerblue-2.0.1/src/gingerblue-main.c}
  
\end{scriptsize}

\subsection{gingerblue-2.0.1/src/gingerblue-main-loop.c}

\begin{scriptsize}
  
  \lstinputlisting[format=C]{gingerblue-2.0.1/src/gingerblue-main-loop.c}
  
\end{scriptsize}

\subsection{gingerblue-2.0.1/src/gingerblue-record.c}

\begin{scriptsize}
  
  \lstinputlisting[format=C]{gingerblue-2.0.1/src/gingerblue-record.c}
  
\end{scriptsize}

\subsection{gingerblue-2.0.1/src/gingerblue-song.c}

\begin{scriptsize}
  
  \lstinputlisting[format=C]{gingerblue-2.0.1/src/gingerblue-song.c}
  
\end{scriptsize}

\subsection{gingerblue-2.0.1/src/gingerblue-studio-config.c}

\begin{scriptsize}
  
  \lstinputlisting[format=C]{gingerblue-2.0.1/src/gingerblue-studio-config.c}
  
\end{scriptsize}

\subsection{gingerblue-2.0.1/src/gingerblue-studio-stream.c}

\begin{scriptsize}
  
  \lstinputlisting[format=C]{gingerblue-2.0.1/src/gingerblue-studio-stream.c}
  
\end{scriptsize}

\chapter{Specification}

GNOME Gingerblue 2.0.1 is specified with the Gingerblue XML meta data.

\chapter{Multiple-Location Audio Recording 1.0}

\section{Gingerblue XML Data Structure}

\noindent The Gingerblue XML data structure contains a ``<gingerblue>'' XML root
node, with ``<musician>'', ``<song>'', ``<instrument>'', ``<line>'',
``<label>'', ``<station>'', ``<filename>'', ``<album>'' and
``<studio>'' subnodes.

\subsection{Example}

\begin{tiny}
\begin{verbatim}
<?xml version='1.0' encoding='UTF-8'?>

<gingerblue version='2.0.1'>
  <musician>Root</musician>
  <song>Song</song>
  <instrument>Guitar</instrument>
  <line>Mic</line>
  <label>GNOME</label>
  <station>localhost</station>
  <filename>Song.ogg</filename>
  <album>GNOME</album>
  <studio>file://localhost/</studio>
</gingerblue>
\end{verbatim}
\end{tiny}

\section{Gingerblue XSPF Playlist}

The Gingerblue Playlist is a subset of XSPF stored default in
\url{$HOME/Music/GNOME.xspf} with a reference to the most current
audio recording.

XPSF (``Spiffy'') was specified by Xiph.org and the specification is
available from \url{http://www.xpsf.org/}

\subsection{Example}

\begin{tiny}
\begin{verbatim}
<?xml version=''1.0'' encoding=''UTF-8''?>
<playlist version=''1'' xmlns=''http://xspf.org/ns/0/''>
  <trackList>
    <track>
      <title>Song_-_2021-10-26T04:39:15Z</title>
      <location>file://perceptron.stream//Users/wilber/Music/Wilber_-_Song_-_2021-10-26T04:39:15Z.ogg</location>
    </track>
  </trackList>
</playlist>
\end{verbatim}
\end{tiny}

\section{Gingerblue HTML 1.0 Document}

\subsection{Example}

\begin{tiny}
\begin{verbatim}
<html>
  <head>
    <title>Wilber - Song</title>
  </head>
  <body>
    <h1>Wilber</h1>
    <h2><a href='http://wiki.gnome.org/Apps/Gingerblue'>Gingerblue 2.0.1</a></h2>
    <p>
      <a href='/home/wilber/Music/Wilber_-_Song_[2021-11-22T21:33:00].ogg'>/home/wilber/Music/Wilber_-_Song_[2021-11-22T21:33:00].ogg</a>
      (Ogg Vorbis, 44.1kHz, Mono)
    </p>
    <p>XSPF: <a href='/home/wilber/Music/GNOME.xspf'>/home/wilber/Music/GNOME.xspf</a></p>
  </body>
</html>
\end{verbatim}
\end{tiny}

\part{Conclusion}                     %% ... or Konklusjon

GNOME Gingerblue 2.0.1 can be configured and compiled with the GNU C
Compiler (\url{GCC.GNU.ORG}), GNU Autoconf and GNU Automake on macOS
11.6 with MacPorts 2.7.1 (\url{MACPORTS.ORG}) and is capable of
recording audio from the built-in microphone on Apple MacBook Air M1
(2020) (\url{APPLE.COM}).

The recording can be achieved manually with the following statements:

\begin{itemize}
\item Install MacPorts 2.7.1 from \url{https://www.macports.org/}
\item Install binary package from macports.org \begin{verbatim}
  sudo port install gingerblue

  gingerblue
\end{verbatim}

\item Install dependencies from macports.org \begin{verbatim}
  sudo port install git desktop-file-utils geoclue2 geocode-glib
  sudo port install glib2 gstreamer1 libxml2 pango
  sudo port install gstreamer1-gst-plugins-base gtk3 
  sudo port install gstreamer1-gst-plugins-bad
  sudo port install gstreamer1-gst-plugins-good
  sudo port install gstreamer1-gst-plugins-ugly zlib xz
  sudo port install adwaita-icon-theme libchamplain
  sudo port install autoconf automake clang-9.0 geoclue2
  sudo port install geocode-glib gnome-common gtk-doc
  sudo port install intltool itstool
  sudo port install p5.28-xml-sax-expat pkgconfig yelp-tools
\end{verbatim}  
\item Install latest source from gitlab.gnome.org \begin{verbatim}
  git clone http://gitlab.gnome.org/ole/gingerblue.git

  cd gingerblue/

  ./configure --prefix=/usr/local

  make

  sudo make install

  [ENTER PASSWORD]

  /usr/local/bin/gingerblue
\end{verbatim}
\end{itemize}

\chapter{Results}                     %% ... or ??

The formal proof is the audio file that was recorded running GNOME
Gingerblue 2.0.1 on Apple MacBook Air M1 (2020) running macOS 11.6
with MacPorts 2.7.1 (\url{MACPORTS.ORG}) at Universitetsbiblioteket,
a public library at University of Oslo and uploaded to
\url{https://www.gingerblue.org/Universitetsbiblioteket.ogg} and it
follows the optimal environment where this thesis and the software
was written and explored.

\chapter{Patents Cited}

\begin{itemize}
\item 341287 Recording and Reproducing Sounds.  Sumner Tainter.  May 4, 1886.
\item 342214 Recording and Reproducing Speech and Other Sounds.  Chichester A. Bell and Sumner Tainter.  May 4, 1886.
\item 661619 Method of Recording and Reproducing Sound or Signals.  Valdemar Poulsen.  November 13, 1900.
\item 836339 Magnetizable Body for the Magnetic Record of Speech.  P.O. Pedersen.  November 20, 1906.
\item 873078 Electromagnet For Telegraphone Purpose.  Peder P. Pedersen and Valdemar Poulsen.  December 10, 1907.
\item 900392 Sound Recording and Reproducing Instruments.  George Kirkegaard.  October 6, 1908.
\item 1142384 Telegraphone. George S. Tiffany.  June 8, 1915.
\item 1213150 Method of Producing Magnetic Sound-Records for Talking-Motion-Picture Films. Henry C. Bullis.  January 23, 1917.
\item 1639060 Magnetic Talking, Dictating, and Like Machine. Gustav Scheel (System-Stille GmbH).  August 16, 1927.
\item 1640881 High Frequency Biasing. W.L. Carlson and G.W. Carpenter.  August 30, 1927.
\item 1883560 Electromagnetic Sound Recording and Reproducing Machine. Harry E. Chipman.  October 18, 1932.
\item 1883561 Magnetic Sound Recording and Reproducing Head. Harry E. Chipman.  October 18, 1932.
\item 2248790 Sound Recording Device. Arnold Stapelfeldt. (C. Lorenz AG).  July 8, 1941.
\item 2264008 Magnetic Sound Recording Device. Arnold Stapelfeldt (C. Lorenz AG).  November 25, 1941.
\item 2351003 Recording and Reproduction of Vibrations. Marvin Camras and William Korzon.  November 18, 1941.
\item 2351007 Magnetic Recording Head. Marvin Camras. June 13, 1944. 2773120 Magnetic Recording of High Frequency Signals Earl E. Masterson (RCA).  December 4, 1956.
\item 2866012 Magnetic Tape Recording and Reproducing System. Charles P. Ginsburg and Shelby F. Henderson, Jr. (Ampex).  December 23, 1958.
\item 2900443 Magnetic Recorder and Reproducer for Video. Marvin Camras.  August 18, 1959.
\item 2900444 Means for Recording and Reproducing Video Signals. Marvin Camras.  August 18, 1959.
\item 2912517 Magnetic Tape Apparatus. Robert Fred Pfost (Ampex).  November 10, 1959.
\item 2912518 Magnetic Tape Apparatus. Alexander R. Maxey (Ampex).  November 10, 1959.
\item 2916546 Visual Image Recording and Reproducing System and Method. Charles P. Ginsburg and Ray M. Dolby (Ampex). December 8, 1959.
\item 2916547 Recording and Reproducing System. Charles P. Ginsburg and Shelby F. Henderson, Jr. (Ampex).  December 8, 1959.
\end{itemize}

\begin{thebibliography}{100}
\bibitem{KernighanRitchie78} Kernighan, Brian W., Ritchie, Dennis M., ``The C programming language'', 1978.
\bibitem{Boney96} Boney, L., Tewfik, A.H., and Hamdy, K.N., ``Digital Watermarks for Audio Signals," \emph{Proceedings of the Third IEEE International Conference on Multimedia}, pp. 473-480, June 1996.
\bibitem{BIOS61} British Intelligence Objectives Subcommittee, p. 61.
\bibitem{PIR} Chignell, Hugh, \emph{Public Issue Radio}, Palgrave Macmillan, Great Britain, 2011.
\bibitem{MG}  Goossens, M., Mittelbach, F., Samarin, \emph{A LaTeX Companion}, Addison-Wesley, Reading, MA, 1994.
\bibitem{HK}  Kopka, H., Daly P.W., \emph{A Guide to LaTeX}, Addison-Wesley, Reading, MA, 1999.
\bibitem{Nmungwun89} Nmungwun, A. F., \emph{Video Recording Technology}, \emph{Lawrence Erlbaum Associates, Publishers}, pp. 8-24, 1989.
\bibitem{Pan} Pan, D., \emph{A Tutorial on MPEG/Audio Compression}, \emph{IEEE Multimedia}, Vol.2, pp.60-74, Summer 1998.
\bibitem{Pulkki18} Pulkki, V., Delikaris-Manias, S., Politis, A., ``Parametric Time-Frequency Domain Spatial Audio'', \emph{IEEE Press}, pp. 3-4
\bibitem{Cox95} Cox, G., ``Pioneering Television News'', \emph{John Libbey}
\end{thebibliography}
\bibliography
\backmatter{}\\

\printbibliography

\chapter*{Application Letter to University of Copenhagen}

\begin{verbatim}
To whom it may concern,

I have studied Computer Science (Object-oriented programming) and
Mathematics (Linear Algebra) at University of Oslo since 1997 in
my home city Oslo.

I have been building a network, maintaining network connectivity at
Fjellbirkeland at University of Oslo since 1998-1999 and worked
at Norwegian Computer Center (NR -- www.nr.no) in 2001-2004.

I have worked on building a commercial 400.000 domain network in
Norway since 2003 (Domainnameshop -- www.domainnameshop.com) with
Ståle Schumacher, Dag Fredrik Øien, and Jan Ingvoldstad.

My plans for advanced studies at University of Copenhagen is to
complete my Bachelor of Science degree at University of Oslo in
2024 with 60 study points of Mathematics-Economics, Mathematics or 
Computer Science at a top Danish university in Copenhagen,
where Hans Christian Ørsted worked on electricity in 1820.

I hope to further perfect my work at the Gingerblue project
(www.gingerblue.org) before June 24, 2024.

Ole Kristian Aamot
www.gnomeradio.org
www.gingerblue.org
www.gnomevoice.org
\end{verbatim}

\noindent GNOME Radio: \url{http://www.oleaamot.no/omu/bachelor/Aamot,2020.pdf}

\noindent Gingerblue: \url{http://www.oleaamot.no/uio/bachelor/Aamot,2022.pdf}

\noindent GNOME Voice: \url{http://www.oleaamot.no/ntnu/bachelor/Aamot,2024.pdf}

\end{document}
