\documentclass[a4paper]{article}

\usepackage{graphicx}
\usepackage{url}

\author{Ole Kr. Aamot\\
  Department of Physics\\
  University of Oslo, Norway}
\title{Immediate Graphical Audio Recording in GNOME Gingerblue 4.0.1 (May 1, 2022)}

\begin{document}

\maketitle

\abstract{\it{New technique for recording immediate audio waves is implemented with GStreamer in GNOME Gingerblue 4.0.1 available from www.gingerblue.org and wiki.gnome.org/Apps/Gingerblue\\
The full paper, Bachelor's thesis report, essay, conference talks and
software downloads and updates for Debian GNU/Linux 11, Fedora 36 and
Ubuntu 22.04 LTS are published on the GNOME Gingerblue Project's
website at www.gingerblue.org and this author's homepage at
oleaamot.no}}\\

Audio waves are air-pressures that provide the physical basis of recording
these waves, discovered by Thomas Alva Edison in 1877.

Audio waves differ in their frequency and wavelength.

With the software GNOME Radio 16 on GNOME 42, audio waves are travelling from
recording stations such as WMBR at M.I.T. in Cambridge, MA in Boston up to
distances larger than five thousand kilometers between Boston and Oslo
where we as listeners on computers in Oslo can listen to sound waves
recorded with a microphone on a computer in Boston between two
continents in the world due to audio waves and the products
from work in electrical engineering.

Nyquist-Shannon Sampling Theorem is named after the American scientists
Harry Nyquist and Claude Shannon and Nyquist's Law states that the
recording frequency must be the double of the desired sampling
frequency to reproduce a human-hearable signal.

145 years ago American engineer Thomas Alva Edison discovered the first 
evidence of the relationship of audio waves with mechanical sampling methods.

On May 1, 2022, based on the previous work towering in electrical
engineering done by Anton Pryima, Wim Taymans, Miguel de Icaza,
Federico Mena, Linus Torvalds, Richard M. Stallman, John von Neumann,
Alan Turing, Thomas Alva Edison, James Clerk Maxwell, Andr\'e-Marie Amp\`ere, Hans
Christian \O{}rsted and Michael Faraday, the immediate graphical
audio recording technique in GNOME Gingerblue 4.0.1 was discovered.

The Immediate Graphical Audio Recording can be recorded and heard in
time-space for a previous audio wave on a GNU/Linux computer from
Hewlett Packard installed with Debian 11, Fedora 36 or Ubuntu 22.04
LTS operating system that is running the gingerblue software
documented in the Bachelor's thesis Public Audio Recording Software
for Recording World Sounds (University of Oslo, May 2022) in the
Bachelor of Science program in Electrical Engineering, Informatics and
Technology at Department of Physics, University of Oslo.

The full paper, Bachelor's thesis report, essay, conference talks and
software downloads and updates for Debian GNU/Linux 11, Fedora 36 and
Ubuntu 22.04 LTS are published on the GNOME Gingerblue Project's
website at www.gingerblue.org and this author's homepage at oleaamot.no

The experiments with the Immediate Graphical Audio Recording technique
in GNOME Gingerblue that was discovered and observed on May 1, 2022,
based on the previous work towering in electrical engineering done by
Anton Pryima, Wim Taymans, Miguel de Icaza, Federico Mena, Linus
Torvalds, Richard M. Stallman, John von Neumann, Alan Turing, Thomas
Alva Edison, James Clerk Maxwell, Andr\'e-Marie Amp\`ere, Hans
Christian \O{}rsted and Michael Faraday, was confirmed on Hewlett
Packard hardware running Fedora Core 36 and Apple MacBook Air M1/8/256
2020) running MacPorts in Oslo, Norway on May 1st, 2022.

See www.gingerblue.org and wiki.gnome.org/Apps/Gingerblue for details on the Immediate Graphical Audio Recording efforts in GNOME Gingerblue.

\end{document}
