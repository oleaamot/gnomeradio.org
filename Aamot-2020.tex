\documentclass[a4paper]{article}

\author{Ole Kr. Aamot\\
  Department of Informatics\\
  University of Oslo, Norway}
\title{Radio flux in GNOME Radio fields confirmed (2020)}

\begin{document}

\maketitle

Electromagnetic waves are disturbances that provide the physical basis for 
light, radio and television waves, as well as infrared, ultraviolet and x 
ray waves discovered by Marie Curie. Visible light, audible radio, 
detectable x rays and other types of electromagnetic waves differ only in 
their frequency and wavelength. Electromagnetic waves as visible light 
from the stars of night has travelled without difficulty across tens or 
hundreds of light-years of (nearly) empty space and audible radio sound 
waves are travelling from stations such as WMBR at M.I.T. in Cambridge, MA 
in Boston up to distances larger than five thousand kilometers between 
Boston and Oslo where we as listeners on computers in Oslo can listen to 
sound waves recorded with a microphone on a computer in Boston between two 
continents in the world due to electromagnetic waves and the products from 
work in electrical engineering. Faraday's Law is named after the English 
scientist Michael Faraday (1791-1867) who first introduced the concept of 
field lines.  He called them "lines of force" (Young/Freedman 2016), but 
the term "field lines" is preferable. An electric field line is an 
imaginary line or curve domain through a region of space so that its 
tangent at any point is in the direction of the electrical-field vector at 
that point. Electric field lines show the direction of the field E at each 
point and each electric field has a certain electric charger flow, a given 
electric flux.

200 years ago Danish scientist Hans Christian \O{}rsted discovered the first 
evidence of the relationship of magnetism to moving 
charges as he found that a compass needle was deflected by a 
current-carrying wire. Amp\`ere's Law, the relation of magnetism to moving 
charges discovered by Andr\'e Amp\`ere, inducing the displacement current 
discovered by James Clerk Maxwell, shows that a time-varying electric 
field acts as a source of magnetic fields, and is formulated in terms of 
the line integral of the magnetic field B around a closed path, denoted by 
Amp\`ere's Law for magnetic flux. A few years later Michael Faraday in 
England and Joseph Henry in the United States discovered that moving a 
magnet near a conducting loop can cause a current in the loop. The mutual 
interaction between the two fields, electric and magnetic fields, is 
summarized in Maxwell's Equations. Faraday's Law tells us a time-varying 
magnetic field acts as a source of electric fields, while Amp\`ere's Law 
shows that a time-varying electric field acts as a source of magnetic 
fields.

On June 12th 2020, based on the previous work towering in 
electrical engineering done by Wim Taymans, Miguel de Icaza, Federico Mena, Linus 
Torvalds, Richard M. Stallman, John von Neumann, Alan Turing, James Clerk 
Maxwell, Andr\'e-Marie Amp\`ere, Hans Christian \O{}rsted and Michael Faraday, the 
GNOME Radio Flux was discovered.

The GNOME Radio Flux can be seen and heard for a radio field line visible on the 
screen of a GNU/Linux computer from Hewlett Packard installed with Debian 10, Fedora 32 or 
Ubuntu 20.04 LTS operating system that is running the 
gnome-internet-radio-locator software documented in the Bachelor's thesis 
Public Internet Radio Client for Accessing Free Audio Maps in Countries 
with Free Speech (Oslo Metropolitan University, June 2020) in Electrical Enginering.

The full paper, Bachelor's thesis report, essay, conference talks and software downloads and updates for Debian GNU/Linux 10, Fedora 32 and Ubuntu 20.04 are published on the GNOME Radio Project's website at www.gnomeradio.org and this author's homepage at home.ifi.uio.no/\textasciitilde{}olekaa/

The experiments with the GNOME Radio Flux that was discovered and
observed on June 12th, 2020, based on the previous work towering in
electrical engineering done by Wim Taymans, Miguel de Icaza, Federico
Mena, Linus Torvalds, Richard M. Stallman, John von Neumann, Alan
Turing, James Clerk Maxwell, Andr\'e-Marie Amp\`ere, Hans Christian
\O{}rsted and Michael Faraday, was confirmed on Hewlett Packard
hardware running Fedora Core 32 in Oslo, Norway on July 3th, 2020.

See www.gnomeradio.org and planet.gnome.org for details on the effort.

\end{document}
